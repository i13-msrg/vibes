\chapter{Related Work}
\label{chapter:relatedWork}
Different related works were already presented in VIBES. The most similar related work is the Bitcoin-Simulator, its architecture and design are different, it can only simulate maximal 6000 nodes and has no transactions. In contrast, VIBES has transactions and unlimited nodes. The two other presented related works are the bitcoin testnet and the back of the envelope approach. Testing and analyzing metrics on the bitcoin testnet is essentially the same as on the public bitcoin network and is almost as resource-intensive. The back of the envelope approach using empirical data to infer certain properties of the network is also impractical due to the very expensive, time-consuming and non-configurable nature \cite{vibes}.

Despite numerous bitcoin simulators existing, since the release of VIBES until the writing of this thesis nothing has been published as similar to VIBES as the Bitcoin-Simulator.

There are some single purpose bitcoin simulators in the research fields of mining \cite{Carlsten:2016:IBW:2976749.2978408} or network \cite{DBLP:conf/im/NeudeckerAH15} \cite{sarrias2015}. The main difference between these simulators and VIBES is the attachment of importance to the user interface. The designs and architectures are also completely different.

To understand the approach of this thesis, an overview over the existing classes and concepts of VIBES is necessary.

For the double-spending attack the research paper  \textit{Analysis of hash rate-based double-spending} is presented and for the flood attack the research paper \textit{Stressing Out: Bitcoin “Stress Testing”}.

\section{VIBES: Fast Blockchain Simulations for Large-scale Peer-to-Peer Networks}
The architecture and design lend itself to being split up into the frontend and the backend. The communication channel between both is HTTP.

\subsection{Frontend}
As JavaScript framework React was chosen for the frontend \cite{react}. The most interesting part of the frontend is the Atomic Design \cite{atomicdesign}. This approach makes it easy to add new components to the user interface or to remove components from pages. The modularity makes the maintainability easy. Atomic design is a philosophy that encourages the composition of entities. There are four of these entities, the smallest components like buttons and inputs are called atoms. Atoms, in turn, assemble molecules, several molecules make up an organism. Finally, several organisms create a page. Actually, several organisms create templates which then create pages, but templates are not used in VIBES. The reason for not using templates could be that this abstraction layer was not needed.

\begin{figure}[!htb]
\centering
\includegraphics[height=0.45\textwidth]{"figures/atomic-design".PNG}
\caption{Atomic Design \cite{atomicdesign}
\label{fig:atomicDesign}}
\end{figure}

\subsection{Backend}
The backend of VIBES has in principle four main types of classes: actions, actors, models, and utils. The following sub-chapters explain them and the Main object.

\subsubsection{Actors and Actions}
\begin{figure}[!htb]
\centering
\includegraphics[height=0.4\textwidth]{"figures/actorModel2".PNG}
\caption{The Actor Model - Usage of message passing avoids locking and blocking
\label{fig:actorModel}}
\end{figure}

VIBES uses the Actor Model \cite{Hewitt:1973:UMA:1624775.1624804}. An Actor is a computational entity, the primary unit of concurrency and actors communicate via messages. There is no shared state and message-first communication, therefore no locks and no blocking exist as shown in Figures \ref{fig:actorModel} and \ref{fig:bestGuess}. This is very important to make sure the simulator is fast.

\begin{figure}[!htb]
\centering
\includegraphics[height=0.4\textwidth]{"figures/actorModel".PNG}
\caption{The Actor Model - Usage of message passing avoids locking and blocking
\label{fig:bestGuess}}
\end{figure}

The implementation in Scala and Akka has five actors with their corresponding actions. Actions describe the methods of Actors. Figure \ref{fig:architecture} shows this architecture.

\begin{figure}[!htb]
\centering
\includegraphics[height=0.5\textwidth]{"figures/vibesArchitecture".PNG}
\caption{VIBES’ Architecture
\label{fig:architecture}}
\end{figure}

\textbf{MasterActor \& MasterActions}: The MasterActor is also called Coordinator, controls the nodes and the execution order in the network. The MasterActor grants permission to nodes to work on a work request. A \textbf{work request} is a piece of computational unit for whose execution the node needs permission, for example, mining a block. If the MasterActor wants to execute a work request with a timestamp in the future, it can \textbf{fast-forward} the entire system to this point in the future.

\textbf{DiscoveryActor \& DiscoveryActions}: The DiscoveryActor is responsible for assigning and updating the neighbours of nodes.

\textbf{NodeActor \& NodeActions}: The NodeActor is the equivalent to a miner in the bitcoin network, interacts with other nodes, blocks, and transactions. The NodeActor is also responsible for the generation of the best guess of a work request. The \textbf{best guess} is, for example, the time in the future when a node wants to mine a block.

\textbf{NodeRepoActor \& NodeRepoActions}: The NodeRepoActor helps the MasterActor, is the repository for all nodes, starts and ends the simulation.

\textbf{ReducerActor \& ReducerActions}: The ReducerActor is called after the simulation to calculate, summarize and prepare the simulation results for the transfer to the user interface.

\begin{figure}[!htb]
\centering
\includegraphics[height=0.5\textwidth]{"figures/workRequest".PNG}
\caption{Nodes voting to fast-forward
\label{fig:workRequests}}
\end{figure}

\subsubsection{Models}
There are five data structures called models.

\textbf{VNode}: VNode is the model of a node in the bitcoin network. It has methods to interact with blocks, transactions, nodes, and the blockchain.

\textbf{VBlock}: VBlock is the model of a block in the bitcoin network, it also has a method to calculate the propagation times of a block.

\textbf{VTransaction}: VTransaction is the model of a transaction in the bitcoin network.

\textbf{VEventTypes}: VEventTypes is a special model to provide the frontend with information about the simulation and events. There are three types: \textit{MinedBlock}, \textit{TransferBlock} and \textit{ReducerResult}.

\textbf{VRecipient}: VRecipient is a very simple model of a event receiver consisting only out of two nodes and a timestamp.

\subsubsection{Utils}
\textbf{VExecution}: Each work request has a \textbf{type}, there are four types which are defined in VExecution. 

\begin{itemize}
\item \textit{MineBlock} is a node's work request to solve the current block. 
\item \textit{IssueTransaction} is a node's work request to create a transaction. 
\item \textit{PropagateTransaction} is a node's work request to propagate transactions.
\item \textit{PropagateOwnBlock/PropagateExternalBlock} is a node's work request to propagate a block to its neighbours.
\end{itemize} 

\textbf{VConf}: The \textbf{configuration parameters} from the user interface are saved in the VConf which can be accessed globally.

\textbf{Joda}: An object for ordering the timestamps.

\subsubsection{Main}
The Main object starts the backend. It gets the configuration parameters from the frontend, saves them in the VConf and prepares the simulation. Then it starts the simulation. After the end of the simulation the Main object provides the simulation results to the frontend.

\section{Analysis of hash rate-based double-spending}
The research paper \textit{Analysis of hash rate-based double-spending} focuses on the quantitative aspects of bitcoin's double-spending prevention and how they relate to attack vectors and their countermeasures. It takes a look at the stochastic processes underlying typical attacks and their resulting probabilities of success. It provides a formula to calculate the success probability of a double-spending attack depending on the attacker's percentage of the network's total hashrate and the confirmations the merchants are waiting for. A formula to calculate the maximal safe transaction value is also provided. Both of these formulas are used in this thesis and the models with their configuration parameters are implemented.

\section{Stressing Out: Bitcoin “Stress Testing”}
The research paper \textit{Stressing Out: Bitcoin “Stress Testing”} is an empirical study of a spam campaign in the summer 2015 that resulted in a DoS attack on bitcoin \cite{DBLP:conf/fc/BaqerHMW16}. The attacker tested several different attack vectors with valid and invalid transactions and even managed to crash over 10\% of all bitcoin nodes at one point. The impacts were measured, for example, the total costs of the attack and the average price increase for transactions are calculated.