\chapter{Background} 
\label{chapter:background}
This chapter will give an overview over the background concepts necessary to understand the following chapters. The nature of a simulator, the concepts of blockchain, bitcoin and various attacks are explained. Unnecessary information for this thesis may be omitted.

\section{Simulation and Simulator}
A simulation is the imitation of the operation of a real-world process or system \cite{bcnn2000}. This requires a model representing the key characteristics, behaviours, and functions of the selected system \cite{simulation}. 

A simulator is a tool that can be manipulated to observe the outcomes of different assumptions or actions. It models the behaviour of a real system with minimum information loss, less used resources and ideally faster than real-time. Simulators are often used for the optimization of systems, studying and gaining insights into the functioning of simulation models.

\section{Bitcoin}
Bitcoin is a purely peer-to-peer electronic currency published in the paper \textit{Bitcoin: A Peer-to-Peer Electronic Cash System} and as open-source software in 2009 by an unknown person or a group of people under the name Satoshi Nakamoto \cite{nakamoto2012bitcoin}. It allows online payments to be sent directly from one party to another without going through a financial institution or the need to trust a third party. Bitcoin not only refers to the currency but is also a currency unit and can be shortened to BTC. Bitcoin uses digital signatures and proof-of-work to prevent double-spending.

\subsection{Double-Spending Problem}
The double-spending problem refers to the problem of electronic cash to prevent some money being spent more than once. Malicious actors can try a double-spending attack to commit fraud. Merchants or users of bitcoin can reduce their double-spending fraud risk by increasing the number of confirmations which they are waiting for \cite{irreversibletransactions}.

In Chapter \ref{subsection:background:alternativhistoryattack} the double-spending attack is explained and in \ref{subsubsection:CalculatingTheMaximalSafeTransactionValue} the maximal safe transaction value is calculated.

\subsection{Proof-of-work and Blockchain}
Digital signatures and proof-of-work provide the main benefit of not requiring a trusted third party to prevent double-spending. The peer-to-peer network timestamps transactions by hashing them into an ongoing chain of hash-based proof-of-work, also called blockchain. This process is called mining. The blockchain can't be changed without redoing the proof-of-work. One single part of this ongoing chain of proof-of-work is called a block. The longest blockchain serves as a proof of the history of transactions and blocks generated by the largest pool of CPU power \cite{nakamoto2012bitcoin}.

VIBES already follows partly bitcoin's protocol \cite{vibes}. This thesis's goal is to implement more features of the protocol.

\subsection{Block}
\begin{figure}[!htb]
\centering
\includegraphics[height=0.3\textwidth]{"figures/Bitcoin_Block_Data".PNG}
\caption{Bitcoin Block Data \cite{bitcoinBlockData}
\label{fig:bitcoinBlockData}}
\end{figure}

In the bitcoin protocol, the blockchain is implemented as a directed tree consisting of blocks \citep{vibes}. As shown in Figure \ref{fig:bitcoinBlockData} each block has certain data like the hash of the previous block, the block timestamp, the transaction root, and the nonce.

The hash references to the block that came immediately before it. It is necessary to establish a chain of blocks.

Each block contains a block timestamp. This timestamp serves as a source of variation for the block hash and makes it also more difficult for an adversary to manipulate the blockchain \cite{blockTime}. 

The transaction root is a Merkle root, it is the hash of all the hashes of all the transactions in the block. With this transaction root, it is possible to securely verify that a transaction has been accepted by the network and get the number of confirmations by downloading just the tiny block headers and Merkle tree, downloading the entire blockchain is unnecessary \citep{merkleRoot}. 

\begin{figure}[!htb]
\centering
\includegraphics[height=0.3\textwidth]{"figures/merkleRoot".PNG}
\caption{Merkle Tree \cite{merkleRoot}
\label{fig:merkleRoot}}
\end{figure}

The nonce is like the block hash a 32-bit field and this value is adjusted by miners to make the hash of the block less than or equal to the current target of the network \cite{nonce}. This is often called a difficult-to-solve mathematical puzzle, the nonce is unique to each block. 

\subsubsection{Block Time and Difficulty}
The current target of the network is related to the difficulty, the difficulty is a measure of how difficult it is to find a hash below a given target \cite{target}. The difficulty adjustment is necessary for the average block time to be close to the target block time. Deviations of the average block time from the target block time have an effect on the stale block rate.

A block in the bitcoin protocol is supposed to be mined every ten minutes. This block time depends on the difficulty and the hash power. If the hash power is increasing, then the block time is lower than ten minutes because the difficulty adjustment is delayed. It is supposed to happen every 2016 blocks or approximately every 14 days.

\subsubsection{Genesis Block}
The first block of a blockchain is called genesis block \cite{genesisBackground}. Genesis is Ancient Greek and means creation or birth. Of course, the genesis block can't contain a hash reference to a previous block.

\subsubsection{Orphan Block and Stale Block}
An orphan block is a block which has no known parent in the currently-longest blockchain. This means a node received a block before its parent block, which could be part of the blockchain. An orphan block is not to be confused with a stale block. A stale block has a known parent but is no longer part of the longest chain \citep{vocabulary}. Figure \ref{fig:staleAndOrphanBlocks} visualizes these definitions.

\begin{figure}[!htb]
\centering
\includegraphics[width=0.8\textwidth]{"figures/staleAndOrphanBlocks".PNG}
\caption{Stale and Orphan Blocks \cite{staleAndOrphanBlocks}
\label{fig:staleAndOrphanBlocks}}
\end{figure}

\subsubsection{Block Size Limit}
Blockchain protocols can have a block size limit which rejects all blocks with a higher block size \cite{blockSizeLimit}. This limits the number of transactions per block depending on the average transaction size.

In the original open-source software, the block size was limited to 32 MiB \cite{weightunits}. In 2010 Satoshi Nakamoto secretly introduced a block size limit of 1 MB. The reason for this secret introduction is assumed to be the protection of the bitcoin network from a DoS attack using blocks of unlimited size. Some nodes would not be willing to accept big blocks and then the chain would split \cite{dinkins}. Until the introduction of SegWit, the maximum size of a Bitcoin block was 1 MB.

\subsection{Transaction}
A transaction is a transfer of BTC that is broadcast to the network and committed into one block \cite{transaction}. If too many transactions are sent, then the non-processed transactions are saved in the transaction pools of the miners. A sender of a transaction has to pay a transaction fee to the miner for him to include the transaction into a block. Otherwise there would be no financial incentive for the miner to include transactions, instead, it would only cost him computation and a miner that ignores transactions would be faster and earn more BTC. Transaction fees are the second financial incentive for miners next to block rewards. A miner can optimize his earned transaction fees by ordering the transactions in his transaction pool by transaction fee divided through transaction size.

% todo: check if you can put the two confirmation figures in the correct order

\begin{figure}[!htb]
\includegraphics[width=\textwidth]{"figures/security".PNG}
\caption{Transaction Security \cite{doublespending2} \label{fig:security}}
\end{figure}

\subsubsection{Confirmations}

A transaction is confirmed as soon as the transaction is part of a block of the blockchain. A transaction confirmed in the most recent blocks can still be removed from the blockchain by creating another longer blockchain. This longer blockchain replaces then the original one. For this reason, merchants wait for a certain number of confirmations. As illustrated in Figure \ref{fig:confirmations}, confirmations are the number of blocks that were created after the block with a transaction. Figure \ref{fig:security} shows requiring more confirmations reduces this risk. Six confirmations is a widespread recommendation and require you to wait on average for one hour to be certain about receiving a transaction. Six transactions mean that even if an attacker owns 20\% of the hash power, he would only have a chance of 2.3\% for a successful double-spending attack.

\begin{figure}[!htb]
\includegraphics[width=\textwidth]{"figures/confirmations".PNG}
\caption{Confirmations \cite{confirmation} \label{fig:confirmations}}
\end{figure}

\subsubsection{Satoshi}
Satoshi is currently the smallest unit of the bitcoin cryptocurrency named after the original creator \cite{satoshi}. One satoshi equals 0.00000001 BTC, 100.000.000 satoshi equals one bitcoin. Transaction fees are most commonly denominated in satoshi.

\subsection{Full Node and Miner}
Full nodes are clients that have validated the whole blockchain self-sufficiently. They enforce all of bitcoin's rules on any received data and can't be cheated through invalid blocks or transactions.

A full node does not need to keep all blockchain data, it can also run in pruning mode \cite{fullNode} \cite{fullNode2}. Pruning mode allows the deletion of all data to make the blockchain size stay under a specified target size.

Also, a miner does not necessarily need the complete blockchain, the miner only needs to have the latest valid block. For example, members of mining pools only need to receive work from the mining pool. A miner is a provider of hash power for proof-of-work. Miners want to earn block rewards and transactions fees by adding blocks to the blockchain.

\subsubsection{Block Rewards and Mining Pools}
Miners are incentivized by block rewards to provide their hashing power. Mining is possible with CPUs, GPUs, application-specific integrated circuits (ASIC) or even a sheet of paper and a pencil, although mining with an ASIC is most profitable \cite{ASIC}. Since it can take years for miners to generate a block, mining pools were created to pool resources of miners together. The reward is then split equally according to their share of contributed work \cite{miningPool}.

\subsection{Segregated Witness}
Segregated Witness (SegWit) is a bitcoin softfork activated on 24\textsuperscript{th} August 2017 as a solution to the scalability problem. After the activation of SegWit, the 1 MB block size limit was replaced with an almost 4 MB big block weight limit \cite{segWit}. This block weight of almost 4 MB is more of theoretical nature, to fill a block with 4 MB it requires all transactions to be very weirdly formatted. The softfork was intended to increase the block capacity, increase the tps and therefore increase scalability. This is achieved by defining a new structure called the witness, which is used to check transaction validity and is committed to a block separately from the transaction Merkle tree. The witness structure is not required to determine transaction effects. This approach achieves great backward compatibility, SegWit-enabled and Non-SegWit bitcoin nodes can work on the same blockchain. One of the points of concern is that SegWit is expected to increase the tps only by a factor between 1.8 to 2.3. The average transaction make-up in 2017 would lead to a block size of 2.3 MB if all transactions were SegWit transactions. But SegWit does not only increase the tps, it also allows other scalability solutions like the Lightning Network to work by adding transaction malleability. Transaction malleability means the signature doesn’t encompass all transaction data and a user could potentially change a transaction ID. Another point of concern is the necessary complex software update \cite{seminarPaper}.

\subsection{Fork}
There are mainly four distinct meanings for fork \cite{Fork}:

There is the chain fork, it occurs when multiple blocks are mined at the same height. Usually, this results in one of the blocks winning and the other blocks are stale blocks.

The softfork is a change to the protocol wherein only previously valid blocks or transactions are made invalid. Softforks are backward compatible, SegWit is an example of a softfork.

Thirdly, there is the hardfork. It makes previously invalid blocks or transactions valid. Hardforks are not backward compatible, Bitcoin Cash is an example.

The (source) code fork is an altcoin that is a derivative of Bitcoin. For example, Litecoin is a code fork of Bitcoin, but neither is a hardfork nor a softfork. The reason for this is Litecoin and Bitcoin do share the same genesis block.

\subsection{Double-Spending Attacks\label{subsection:background:alternativhistoryattack}}
The entirety of bitcoin's system of blockchain, mining, proof-of-work, difficulty etc. exist to make the history of transactions irreversible and to solve the double-spending problem. When bitcoin is used correctly, the transactions on the blockchain are irreversible and final \cite{irreversibletransactions}. There are still scenarios to successfully spend bitcoin twice. These double-spending attacks depend on the number of confirmations a merchant/transaction receiver is waiting for and the hash power of the attacker. By redoing the proof-of-work and creating the longest blockchain an attacker can attempt a double-spending attack. An attacker can also abuse low-security confirmation and network settings.

\subsubsection{Race Attack\label{subsubsection:raceAttack}}
A merchant or a transaction receiver operating his own bitcoin node who accepts a payment on seeing the transaction status "0/unconfirmed" is at risk of a race attack and double-spending fraud. A malicious actor could send a transaction directly to the node of the transaction receiver and a conflicting transaction with a higher transaction fee to the rest of the bitcoin network with a different transaction receiver. The transaction with the higher transaction fee is more likely to be mined into a block, this also depends on the number of pending transactions.

According to the research paper \textit{Two Bitcoins at the Price of One} \cite{Karame12twobitcoins} an attacker has a high degree of success in performing a race attack 

As precautions, a transaction receiver can disable incoming connections and can connect to only well-connected nodes to lessen the risk of a race attack, but the risk can't be eliminated. This is another reason why waiting for six confirmations is recommended. There is a theoretical solution to enable fast and secure bitcoin transactions, alerts in case of double-spending fraud suspicions \cite{Karame12twobitcoins}. But there is no adaptation yet and making fast and secure bitcoin transactions are still very difficult.

\subsubsection{Finney Attack\label{subsubsection:finneyAttack}}
The Finney attack is another attack on transaction receivers who accept payments on the transaction status "0/unconfirmed". The Finney attack requires hash power. The attacker mines a block and includes a transaction from address A to address B, which he both controls. Now he sends a transaction from address A to transaction receiver's address C, the transaction receiver thinks the transaction is final after receiving the transaction status "0/unconfirmed". But the attacker broadcasts his mined block afterwards. His transaction to address B takes precedence over the transaction to address C. 

\subsubsection{Vector76 Attack\label{subsubsection:vector76Attack}}
The Vector76 attack is a complex attack combining the race and the Finney attack, which can even reverse transactions included in the latest block. The malicious miner has to find an opportunity worth more than the current block reward to make the attack profitable, because the attack requires the miner to intentionally let a mined block become a stale block.

The malicious miner solves a block and includes transaction A sending BTC to a victim. Instead of broadcasting the solved block, the miner broadcasts the transaction A via node A (connected directly to the victim) and a second transaction B (via a well-connected) node B. Transaction B does not send money to the victim but to the malicious miner. Eventually, another miner will solve a block and include either transaction A or transaction B. The connectivity of node B and the transaction fees of transaction B make it more likely to take precedence over transaction A. The malicious miner sends his own solved block with transaction A via node A to the victim after seeing the solved block with transaction B via node B. At this moment the victim only sees transaction A in the block mined by the malicious miner, assumes everything is correct and does a beneficial action for the malicious actor. Now there are effectively two branches of the blockchain since it is likely that the majority of the hashing power has the block with transaction B, it will create a new child block and therefore be the longest blockchain and erase the other branch with transaction A. If transaction A gets propagated faster, there should be no/minimal loss for the malicious actor, except for the costs to produce a stale block. In case the block with transaction A gets propagated faster, the malicious actor earns the block reward and transaction fees.

A Vector 76 attack has a very high chance to be profitable, but it is very unlikely to find such an opportunity.

\subsubsection{Alternative History Attack\label{subsubsection:alternativeHistoryAttack}}
Figure \ref{fig:doublespending} visualises the concept behind the alternative history and the majority attack. In contrast to the previous attacks, the concept behind these attacks is more well-known.

\begin{figure}[!htb]
\includegraphics[width=\textwidth]{"figures/doublespending".PNG}
\caption{Double-Spending Attack \cite{doublespending2} \label{fig:doublespending}}
\end{figure}

Comparable to the Finney and Vector76 attacks, the attacker needs a significantly higher hashrate. The alternative history attack can even work if the transaction receiver waits for some confirmations. The higher the attacker's percentage of the network's total hashrate, the more confirmations are needed to prevent double-spending.

Like the name alternative history attack implies, the miner starts working on an alternative history, his own private blockchain after his transactions were included in a block A. Multiple transactions and double-spending attempts targeting different victims at the same time make the attack more profitable. The private blockchain has the same parent block as block A and the first block includes the fraudulent double-spending transactions. After the victims waited for their number of confirmations, accepted the transactions and did something beneficial for the attacker, the attacker makes his private blockchain public as soon as it is longer than the original blockchain and creates hereby an alternative history.
In case of success, the attacker regains his spent bitcoins and receives beneficial actions from the victims.
In case of failure, the attacker has to bear the hashrate costs and pay for the bought goods or services.

The success probabilities of an alternative history attack depending on the attacker's hashrate and the transaction receiver's number of confirmations are displayed in the Table \ref{fig:doubleSpend}.

\subsubsection{Maximal Safe Transaction Value}
The maximal safe transaction value is the value a participant of the bitcoin network can send safely depending on the success probability of the double-spending attack and other variables like the block reward as can be seen in the Formula \eqref{eq:maximalSafeTransactionValue}. The maximal safe transaction values are shown in Table \ref{fig:maximalSafeTransactionValue}.

\subsubsection{Majority Attack\label{subsubsection:majorityAttack}}
The majority attack is also called 51\% attack. The concept behind the alternative history attack and the majority attack is the same, the difference is the attacker's percentage of the total hash power of the network. For a majority attack equal to or more than 50\% of the total network's hashrate is necessary. A majority attack has a probability of 100\% to succeed, no amount of confirmations can prevent this attack. With at least 50\% of the hashrate the attacker can work secretly until his private blockchain is longer.

\subsubsection{Economic Majority\label{subsubsection:economicMajority}}
For an attacker, a majority attack on bitcoin can be catastrophic due to the effects on the market and the attacker's very high commitment to ASIC mining hardware \cite{economicMajority}. A miner with more than 50\% of the hashrate is therefore incentivized to calm down the market and to reduce his mining power and abstain from attacking to protect the mining hardware.

Altcoins of bitcoin or cryptocurrencies, for which a cryptocurrency with a similar algorithm and higher hashrate exists, are at risk from majority attacks.

\subsection{Transaction Spam Attack}
In July 2018 the Ethereum Network was affected by transaction spam or also called flood attack. In such a flood attack, the attacker, in principle, trades their own cryptocurrency for increased transaction costs for everyone by only using intended functionality and valid transactions. Vitalik Buterin tweeted about this attack \cite{tweet}, the tweet can be seen in Figure \ref{fig:transactionSpam}. This transaction spam is also possible in the bitcoin network. Interesting research questions arise about the costs which a malicious actor has to pay to make the bitcoin network unusable or too uneconomical to use for certain use-cases. Such an attack is only limited on the attacker's number of bitcoin and depends on the target transaction fee, the duration of the attack and the scalability of the network. An attacker with a certain amount of money can make blockchains for other users for a certain time unusable.

There are also other definitions for transaction spam or flood attacks. In this thesis, we assume a transaction spam attack only uses valid transactions.

\begin{figure}[!htb]
\centering
\includegraphics[height=0.3\textwidth]{"figures/tweet".PNG}
\caption{Screenshot Tweet from Vitalik Buterin about Transaction Spam
\label{fig:transactionSpam}}
\end{figure}