\begin{appendices}
\addtocontents{toc}{\protect\setcounter{tocdepth}{0}}
\chapter{Appendix}

\section{Installation and Usage Instructions}
\subsection{Installation Frontend}
\subsubsection{Install npm}
You need to install the npm package manager as documented: \url{https://www.npmjs.com/}

\subsubsection{Install yarn}
Then install yarn: \url{https://yarnpkg.com/lang/en/docs/install/}

\subsubsection{Install package.json dependencies}
After the installation is done you can navigate to the frontend folder of the project and run \textit{yarn install}.

\subsection{Installation Backend}
\subsubsection{SBT}
The only thing you need to install is SBT (interactive scala build tool) as documented: \url{https://www.scala-sbt.org/}

\subsection{Run}
\subsubsection{Windows}
Navigate to the root folder and run \textit{start frontend and backend.bat} to start the frontend and backend server. Go to \url{http://localhost:8080/} with your browser.

\subsubsection{Linux}
Run \textit{yarn dev} to start the frontend, run \textit{sbt server/run} to start the backend. 

\subsection{Inspecting the output of the backend}
Start your backend. Then enter the following URL into the browser of your choice after you have adjusted the timestamp of the simulateUntil parameter.

\begin{minipage}{\linewidth}
\begin{lstlisting}[style=batch]
http://localhost:8082/vibe?strategy=BITCOIN_LIKE_BLOCKCHAIN&simulateUntil=1534457813308&blockTime=600&numberOfNeighbours=4&numberOfNodes=20&neighboursDiscoveryInterval=3000&latency=900&transactionSize=1000&maxBlockSize=50&throughput=50&transactionWeight=2000&maxBlockWeight=200000&networkBandwidth=1&transactionPropagationDelay=150&hashRate=40&confirmations=2&transactionFee=0
\end{lstlisting}
\end{minipage}

The browser shows the simulation results that are fetched by the frontend from the backend.

\section{Source code}
Available at \url{https://github.com/i13-msrg/vibes/tree/FabianSchuessler}.

\section{Simulations}
The simulateUntil parameter of the following URLs to reproduce the evaluation results has to be adjusted. New features might make additional parameters necessary.

\subsubsection{Expected and Simulated Success Probability of Double Spending\label{evalDoubleSpending}}
\begin{minipage}{\linewidth}
\begin{lstlisting}[style=batch]
ECHO Start of Loop

FOR /L %%i IN (1,1,100) DO (
  ECHO %%i
  start chrome "http://localhost:8082/vibe?blockTime=567&numberOfNeighbours=4&numberOfNodes=20&simulateUntil=1531411943382&transactionSize=1&throughput=105&latency=900&neighboursDiscoveryInterval=3000&maxBlockSize=100&maxBlockWeight=4000&networkBandwidth=1&strategy=BITCOIN_LIKE_BLOCKCHAIN&transactionPropagationDelay=150&hashRate=30&confirmations=6"
  timeout /t 40
)
\end{lstlisting}
\end{minipage}

\subsubsection{Expected and Simulated Transactions per Second - 6 hours\label{6hoursURL}}

\begin{minipage}{\linewidth}
\begin{lstlisting}[style=batch]
http://localhost:8082/vibe?blockTime=567&numberOfNeighbours=4&numberOfNodes=20&simulateUntil=1531752759474&transactionSize=1&throughput=105&latency=900&neighboursDiscoveryInterval=3000&maxBlockSize=100&maxBlockWeight=4000&networkBandwidth=1&strategy=BITCOIN_LIKE_BLOCKCHAIN&transactionPropagationDelay=150&hashRate=0&confirmations=4
\end{lstlisting}
\end{minipage}

\subsubsection{Expected and Simulated Transactions per Second - 48 hours\label{48hoursURL}}

\begin{minipage}{\linewidth}
\begin{lstlisting}[style=batch]
http://localhost:8082/vibe?blockTime=567&numberOfNeighbours=4&numberOfNodes=20&simulateUntil=1531905360000&transactionSize=1&throughput=105&latency=900&neighboursDiscoveryInterval=3000&maxBlockSize=100&maxBlockWeight=4000&networkBandwidth=1&strategy=BITCOIN_LIKE_BLOCKCHAIN&transactionPropagationDelay=150&hashRate=0&confirmations=4
\end{lstlisting}
\end{minipage}

\section{Tables}

\begin{figure}[!htb]
\includegraphics[width=\textwidth]{"figures/Testing/The probability of a successful double spend".PNG}
\caption{The probability of a successful double spend, as a function of the attacker's hashrate \textit{q} and the number of confirmations \textit{n}. The abscissa shows the confirmations \textit{n} and the ordinate shows the attacker's hashrate \textit{q}. \cite{doublespending}\label{fig:doubleSpend}}
\end{figure}

\begin{figure}[!htb]
\includegraphics[width=\textwidth]{"figures/Testing/maximalSafeTransactionValue".PNG}
\caption{The maximal safe transaction value, in BTC, as a function of the attacker's hashrate \textit{q} and the number of confirmations \textit{n}. The abscissa shows the confirmations \textit{n} and the ordinate shows the attacker's hashrate \textit{q}. \cite{doublespending}\label{fig:maximalSafeTransactionValue}}
\end{figure}

\clearpage

\begin{figure}[!htb]
\centering
\includegraphics[width=1\textwidth]{"figures/Testing/AttackUndecided".PNG}
\caption{Screenshot Block Tree, Branch Selection, and Attack Summary - Attack undecided
\label{fig:AttackUndecided}}
\end{figure}

\begin{figure}[!htb]
\centering
\includegraphics[width=1\textwidth]{"figures/consoleBackend".PNG}
\caption{Screenshot Backend Console
\label{fig:simulation}}
\end{figure}

\begin{figure}[p]
\centering
\includegraphics[width=1\textwidth]{"figures/bbssOnAWS".PNG}
\caption{Screenshot BBSS on AWS
\label{fig:BBSSonAWS}}
\end{figure}

\end{appendices}